\documentclass[cjk,slidestop,mathserif,hyperref={CJKbookmarks=true}]{beamer}
\usepackage{CJK}
%\usepackage[space,noindent]{ctex}
\usepackage{beamerthemesplit} 
\usepackage{graphicx}
\usepackage{listings}
%\usepackage{CJKutf8}
\usetheme{Warsaw}
\usecolortheme{crane}
\usecolortheme{seagull}

%\useinnertheme[shadow]{rounded}

\usepackage{beamerthemeshadow}
\setbeamertemplate{blocks}[rounded][shadow=true]

\setbeamertemplate{footline}[frame number]

\lstset{
    language=c,
    basicstyle=\ttfamily,
    keywordstyle=\color{blue}\ttfamily,
    stringstyle=\color{red}\ttfamily,
    commentstyle=\color{green}\ttfamily,
    basicstyle=\footnotesize,
}

\begin{document}
\begin{CJK*}{UTF8}{gkai}
%\CJKtilde\CJKcaption{GB}

%\defverbatim{\Lst}{%
\title{循环控制}	%标题
\author{Xuehui Song}	%作者
\institute{ECE, PKUSZ} %机构
%\logo{\includegraphics[width=12mm]{pku2.png}} %logo
%\usebackgroundtemplate{\includegraphics[width=\paperwidth]{background.jpg}}

\frame{\titlepage}
 
\frame{\tableofcontents}

\frame{
    \frametitle{概述}
    \begin{definition}
        循环结构是程序中一种很重要的结构。其特点是,在给定条件成立时,反复执行某程序段,
        直到条件不成立为止。给定的条件称为循环条件,反复执行的程序段称为循环体。
        C语言提供了多种循环语句,可以组成各种不同形式的循环结构。
        \begin{itemize}
            \item 用 goto 语句和 if 语句构成循环;
            \item 用 while 语句;
            \item 用 do-while 语句;
            \item 用 for 语句;
        \end{itemize}
    \end{definition}
}

\section{基本循环语句}
\subsection{while语句}
\begin{frame}[fragile]
    \frametitle{while语句}
    \begin{columns}
        \column{6cm}
            \begin{definition}
                while 语句的一般形式为:\\
                \quad\textbf{while(表达式)\\ \qquad 语句}
            \end{definition}
            其中表达式是循环条件,语句为循环体。
            while 语句的语义是:计算表达式的值,当值为真(非 0)时, 执行循环体。
        \column{5cm}
            \begin{figure}
                \includegraphics[width=4.5cm]{while.pdf}
            \end{figure}


    \end{columns}
\end{frame}


\begin{frame}[fragile]
    \frametitle{while例子}
    \begin{columns}
    \column{6cm}
    用while语句求$\sum\limits_{i=1}^{n}i^3$
    \begin{lstlisting}[language=c]
#include <stdio.h>
#include <stdlib.h>
int main() {
    int n, i = 0, sum = 0;
    scanf("%d", &n);
    while(i <= n) {
        sum += i * i * i;
        ++ i;
    }
    printf("sum=%d\n", sum);
    return 0;
}
    \end{lstlisting}
    \column{5cm}
        \begin{figure}
            \includegraphics[width=4.5cm]{while_e.pdf}
        \end{figure}
    \end{columns}
\end{frame}

\subsection{do-while}
\begin{frame}[fragile]
    \frametitle{do-while}

    \begin{columns}
        \column{6cm}
        \begin{definition}
            do-while 语句的一般形式为: \\
            \textbf{\quad do \\
                \qquad 语句 \\
                \quad while(表达式);}
        \end{definition}
do-while先执行循环中的语句,然后再判断表达式是否为
真, 如果为真则继续循环;如果为假, 则终止循环。因此, do-while 循环至少要执行一次
循环语句。其执行过程可用右图表示。
        \column{5cm}
            \begin{figure}[!t]
                \includegraphics[width=4.5cm]{do_while.pdf}
            \end{figure}
    \end{columns}
\end{frame}

\begin{frame}[fragile]
    \frametitle{do-while例子}

    用do-while语句求$\sum\limits_{i=1}^{n}i^3$
    \begin{lstlisting}[language=c]
#include <stdio.h>
#include <stdlib.h>
int main() {
    int n, i = 0, sum = 0;
    scanf("%d", &n);
    do {
        sum += i * i * i;
        ++i;
    } while(i <= n);
    printf("sum=%d\n", sum);
    return 0;
}
    \end{lstlisting}
\end{frame}

\subsection{for语句}
\begin{frame}[fragile]
    \frametitle{for语句}
    \begin{columns}
        \column{6.5cm}
            \begin{definition}
                在 C 语言中,for 语句使用最为灵活,可以取代 while 语句。它的一般形式为:\\
                \textbf{
                \quad for(表达式 1;表达式 2;表达式 3) \\
                \qquad 语句}
            \end{definition}
            它的执行过程如下:
            \begin{enumerate}
                \item 先求解表达式 1。
                \item 求解表达式 2,若为真(非 0),则执行指定的内嵌语句,
                    然后执行第3步;
                    若为假(0),则结束循环
                \item 求解表达式3。
                \item 转回上面第2步继续执行。
            \end{enumerate}
        \column{4.5cm}
            \begin{figure}[!t]
                \includegraphics[width=4cm]{for.pdf}
            \end{figure}
    \end{columns}
\end{frame}

\begin{frame}[fragile]
    \frametitle{for例子}

    用for语句求$\sum\limits_{i=1}^{n}i^3$
    \begin{lstlisting}[language=c]
#include <stdio.h>
int main() {
    int n, i, sum;
    scanf("%d", &n);
    for (i = 0, sum = 0; i <= n; ++i) {
        sum += i * i * i;
    }
    printf("sum=%d\n", sum);
}
    \end{lstlisting}
\textbf{for(循环变量赋初值;循环条件;循环变量增量) \\ \qquad 语句}

\end{frame}

\begin{frame}[fragile, allowframebreaks]
    \frametitle{for注意事项}
    \begin{enumerate}
        \item for 循环中的“表达式 1(循环变量赋初值)”、“表达式 2(循环条件)”和
        “表达式 3(循环变量增量)”都是选择项, 即可以缺省,但“;”不能缺省
        \item 省略了“表达式 1(循环变量赋初值)”,应在for语句之前给循环变量赋初值。
            \begin{lstlisting}[language=c]
    int n, i = 0, sum = 0;
    for (; i <= n; ++i) {
        sum += i * i * i;
    }
            \end{lstlisting}
        \item 省略了“表达式 2(循环条件)”, 默认循环条件始终为真,则不做其它处理时便成为死循环。
            \begin{lstlisting}[language=c]
    for (i = 0; ; ++i) {
        sum += i * i * i;
    }
            \end{lstlisting}
            修改为:
            \begin{lstlisting}[language=c]
    for (i = 0; ; ++i) {
        if (i > n) break;
        sum += i * i * i;
    }
            \end{lstlisting}
        \item 省略了“表达式 3(循环变量增量)”, 则不对循环控制变量进行操作,这时可在语句体中
加入修改循环控制变量的语句。
            \begin{lstlisting}[language=c]
    for (i = 0; i <= n;) {
        sum += i * i * i;
        i = i + 1;
    }
            \end{lstlisting}
        \item 省略了“表达式 1(循环变量赋初值)”和“表达式 3(循环变量增量)”,只保留“表达式 2(循环条件)”。
            \begin{lstlisting}[language=c]
    for (; i <= n;) {
        sum += i * i * i;
        i = i + 1;
    }
            \end{lstlisting}
            相当于:
            \begin{lstlisting}[language=c]
    while (i <= n) {
        sum += i * i * i;
        i = i + 1;
    }
            \end{lstlisting}
        \item 3 个表达式都可以省略。
            \begin{lstlisting}[language=c]
    for (; ; ) {
        sum += i * i * i;
    }
            \end{lstlisting}
            相当于:
            \begin{lstlisting}[language=c]
    while (1) {
        sum += i * i * i;
    }
            \end{lstlisting}
    \end{enumerate}
\end{frame}

\section{扩展}
\subsection{循环嵌套}

\begin{frame}[fragile]
    \frametitle{循环嵌套}
    \begin{definition}
        一个循环体内又包含一个完整的循环结构,称为循环的嵌套。内嵌的循环中还可以嵌套循环,这就是多层循环。
    \end{definition}
    输出99乘法表
        \begin{lstlisting}[language=c]
#include <stdio.h>
int main() {
    int i, j;
    for (i = 1; i <= 9; ++i) {
        for (j = 1; j <= 9; ++j) {
            printf("%5d", i * j);
        }
        printf("\n");
    }
}
        \end{lstlisting}
\end{frame}

\subsection{几种循环的比较}
\begin{frame}[fragile]
    \frametitle{几种循环的比较}
    \begin{itemize}
        \item 四种循环都可以用来处理同一个问题,一般可以互相代替。但一般不提倡用 goto 型循
环。
        \item while 和 do-while 循环,循环体中应包括使循环趋于结束的语句。for 语句功能最强。
        \item 用 while 和 do-while 循环时,循环变量初始化的操作应在 while 和 do-while 语句之前完
成,而 for 语句可以在表达式 1 中实现循环变量的初始化。
    \end{itemize}
\end{frame}

\section{break和continue}
\subsection{break语句}
\begin{frame}[fragile]
    \frametitle{break语句}
    \begin{definition}
        break 语句通常用在循环语句和开关语句中。
        \begin{enumerate}
            \item 当 break 用于开关语句 switch 中时,可使程序跳出 switch 而执行 switch 以后的语句
            \item 当 break 语句用于 do-while、for、while 循环语句中时,可使程序终止循环而执行循环后面的语句, 通常 break 语句总是与 if 语句联在一起。即满足条件时便跳出循环。
        \end{enumerate}
    \end{definition}
\end{frame}

\subsection{continue}
\begin{frame}[fragile]
    \frametitle{continue语句}
    \begin{definition}
    continue 语句的作用是跳过循环本中剩余的语句而强行执行下一次循环。continue 语
句只用在 for、while、do-while 等循环体中,常与 if 条件语句一起使用。
    \end{definition}
    求1到10中,不是3倍数的数的阶乘的和,也就是$1!+2!+4!+\dots+10!$
        \begin{lstlisting}[language=c]
#include <stdio.h>
int main() {
    int sum = 0, fac, i, j;
    for(i = 1; i <= 10; ++i) {
        if (i % 3 == 0) continue;
        fac = 1;
        for(j = 1; j <= i; ++j) fac = fac * j;
        sum += fac;
    }
}
        \end{lstlisting}
\end{frame}

\subsection{循环例子}
\begin{frame}[fragile]
    \frametitle{例子}
    \begin{enumerate}
        \item 用$\pi \approx 1 - \dfrac{1}{3} + \dfrac{1}{5}-\dfrac{1}{7}+\dots$公式求$\pi$的近似值,直到某一项的绝对值小于$10^{-8}$为止
        \item 输入n,统计1到n中回文数字的个数,所谓回文数就是从左往右和从右往左看是一样的数,如33,151,2442等
        \item 输入n,统计1到n中素数的个数
        \item NOIP提高组2005 \quad 谁拿了最多奖学金 \\
        https://vijos.org/p/1001
        \item NOIP提高组2004 \quad 津津的储蓄计划 \\
        https://vijos.org/p/1096
    \end{enumerate}
\end{frame}



\frame{
    \frametitle{Questions and answers}
    \begin{figure}
        \includegraphics[width=6cm]{faq.jpg}
    \end{figure}
}


\end{CJK*}
\end{document}
